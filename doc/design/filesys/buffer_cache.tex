\section*{BUFFER CACHE}

\begin{aspect}{DATA STRUCTURES}
  \begin{qc}
    C1: Copy here the declaration of each new or changed \lstinline{struct} or member,
    global or static variable, \lstinline{typedef}, or enumeration.
    Identify the purpose of each in 25 words or less.
  \end{qc}
\end{aspect}


\begin{aspect}{ALGORITHMS}
  \begin{qc}
    C2: Describe how your cache replacement algorithm chooses a cache
    block to evict.
  \end{qc}
  \begin{qc}
    C3: Describe your implementation of write-behind.
  \end{qc}
  \begin{qc}
    C4: Describe your implementation of read-ahead.
  \end{qc}
\end{aspect}

\begin{aspect}{SYNCHRONIZATION}
  \begin{qc}
    C5: When one process is actively reading or writing data in a
    buffer cache block, how are other processes prevented from evicting
    that block?
  \end{qc}
  \begin{qc}
    C6: During the eviction of a block from the cache, how are other
    processes prevented from attempting to access the block?
  \end{qc}
\end{aspect}

\begin{aspect}{RATIONALE}
  \begin{qc}
    C7: Describe a file workload likely to benefit from buffer caching,
    and workloads likely to benefit from read-ahead and write-behind.
  \end{qc}
\end{aspect}