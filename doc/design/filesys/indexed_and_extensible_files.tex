\section*{INDEXED AND EXTENSIBLE FILES}

\begin{aspect}{DATA STRUCTURES}
  \begin{qc}
    A1: Copy here the declaration of each new or changed \lstinline{struct} or member,
    global or static variable, \lstinline{typedef}, or enumeration.
    Identify the purpose of each in 25 words or less.
  \end{qc}
  \begin{qc}
    A2: What is the maximum size of a file supported by your inode
    structure?  Show your work.
  \end{qc}
\end{aspect}

\begin{aspect}{SYNCHRONIZATION}
  \begin{qc}
    A3: Explain how your code avoids a race if two processes attempt to
    extend a file at the same time.
  \end{qc}
  \begin{qc}
    A4: Suppose processes A and B both have file F open, both
    positioned at end-of-file.  If A reads and B writes F at the same
    time, A may read all, part, or none of what B writes.  However, A
    may not read data other than what B writes, e.g. if B writes
    nonzero data, A is not allowed to see all zeros.  Explain how your
    code avoids this race.
  \end{qc}
  \begin{qc}
    A5: Explain how your synchronization design provides "fairness".
    File access is "fair" if readers cannot indefinitely block writers
    or vice versa.  That is, many processes reading from a file cannot
    prevent forever another process from writing the file, and many
    processes writing to a file cannot prevent another process forever
    from reading the file.
  \end{qc}
\end{aspect}

\begin{aspect}{RATIONALE}
  \begin{qc}
    A6: Is your inode structure a multilevel index?  If so, why did you
    choose this particular combination of direct, indirect, and doubly
    indirect blocks?  If not, why did you choose an alternative inode
    structure, and what advantages and disadvantages does your
    structure have, compared to a multilevel index?
  \end{qc}
\end{aspect}
