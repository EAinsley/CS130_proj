\section*{SUBDIRECTORIES}

\begin{aspect}{DATA STRUCTURES}
	\begin{qc}
		B1: Copy here the declaration of each new or changed \lstinline{struct} or member,
		global or static variable, \lstinline{typedef}, or enumeration.
		Identify the purpose of each in 25 words or less.
	\end{qc}
	In \lstinline{threads/thread.h}, we added the following fields for TCB strcuture:
	\begin{lstlisting}
struct dir *working_directory; // The userprog current working directory (cwd), NULL->root
	\end{lstlisting}
\end{aspect}

\begin{aspect}{ALGORITHMS}
	\begin{qc}
		B2: Describe your code for traversing a user-specified path.
		How do traversals of absolute and relative paths differ?
	\end{qc}
	This can be found in the \lstinline{struct dir* dir_open_path (const char *path)} function in \lstinline{filesys/directory.c}.
	\begin{enumerate}
		\item Reject invalid path: \lstinline{NULL}, with components longer than \lstinline{14} characters.
		\item If the given path starts with an \lstinline{/}, then it is an absolute path.
		      We start from the root directory.
		      Otherwise, the path is relative, we start from the CWD of the userprog process.
		\item Use \lstinline{strtok_r} to extract every path component.
		\item Every time a new path component $x$ is found, we lookup in the current directory for the name $x$.
		\item If the \lstinline{dir_entry} is an directory, then we recurse into $x$. Otherwise, we simply stop the process.
		\item If we stop before all the path components are processed, then the path is invalid.
		      Otherwise, return the file or directory inode found.
	\end{enumerate}
\end{aspect}

\begin{aspect}{SYNCHRONIZATION}
	\begin{qc}
		B4: How do you prevent races on directory entries?  For example,
		only one of two simultaneous attempts to remove a single file
		should succeed, as should only one of two simultaneous attempts to
		create a file with the same name, and so on.
	\end{qc}
	We have a file system level lock.
	\textcolor{red}{TODO}.

	\begin{qc}
		B5: Does your implementation allow a directory to be removed if it
		is open by a process or if it is in use as a process's current working directory?
		If so, what happens to that process's future file system operations?
		If not, how do you prevent it?
	\end{qc}
	No.
	We first open the directory to be removed and check if the inode open count is 1 i.e., no other thread is accesssing the directory.
	After that, we call \lstinline{inode\_remove} and \lstinline{inode\_close} to remove it from disk.
\end{aspect}


\begin{aspect}{RATIONALE}
	\begin{qc}
		B6: Explain why you chose to represent the current directory of a
		process the way you did.
	\end{qc}
	We included a \lstinline{struct dir *cwd} field in PCB, instead of having a path string.\\
	Not to use a string avoids tedious string concatenation and spliting process.
	This also makes parent directory lookup much eaiser.
\end{aspect}
