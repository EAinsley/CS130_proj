\section{SYSTEM CALLS}
\begin{aspect}{DATA STRUCTURES}
	\begin{qc}
		B1: Copy here the declaration of each new or changed
		`struct' or `struct' member, global or static variable, `typedef', or enumeration.
		Identify the purpose of each in 25 words or less.
	\end{qc}
	\TODO{code change for syscall}

	\begin{qc}
		B2: Describe how file descriptors are associated with open files.
		Are file descriptors unique within the entire OS or just within a single process?
	\end{qc}
	\TODO{file descriptor ownership}
\end{aspect}

\begin{aspect}{ALGORITHMS}
	\begin{qc}
		B3: Describe your code for reading and writing user data from the kernel.
	\end{qc}
	\TODO{read/write syscall implementation}

	\begin{qc}
		B4: Suppose a system call causes a full page (4,096 bytes) of data to be copied from user space into the kernel.
		What is the least and the greatest possible number of inspections of the page table (e.g. calls to \lstinline{pagedir_get_page()}) that might result?
		What about for a system call that only copies 2 bytes of data?
		Is there room for improvement in these numbers, and how much?
	\end{qc}
	\TODO{reducing page table lookup}

	\begin{qc}
		B5: Briefly describe your implementation of the \lstinline{wait} system call
		and how it interacts with process termination.
	\end{qc}
	\TODO{wait syscall impl}

	\begin{qc}
		B6: Any access to user program memory at a user-specified address can fail due to a bad pointer value.
		Such accesses must cause the process to be terminated.
		System calls are fraught with such accesses,
		e.g. a \lstinline{write} system call requires reading the system call number from the user stack,
		then each of the call's three arguments, then an arbitrary amount of user memory, and any of these can fail at any point.
		This poses a design and error-handling problem: how do you best avoid obscuring the primary function of code in a morass of error-handling?
		Furthermore, when an error is detected, how do you ensure that all temporarily allocated resources (locks, buffers, etc.) are freed?
		In a few paragraphs, describe the strategy or strategies you adopted for managing these issues.
		Give an example.
	\end{qc}
	\TODO{syscall invalid memory handling}
\end{aspect}

\begin{aspect}{SYNCHRONIZATION}
	\begin{qc}
		B7: The \lstinline{exec} system call returns \lstinline{-1}
		if loading the new executable fails, so it cannot return before the new executable has completed loading.
		How does your code ensure this?
		How is the load success/failure status passed back to the thread that calls \lstinline{exec}?
	\end{qc}
	\TODO{exec syscall load fail handling}

	\begin{qc}
		B8: Consider parent process \lstinline{P} with child process \lstinline{C}.
		How do you ensure proper synchronization and avoid race conditions
		when \lstinline{P} calls \lstinline{wait(C)} before \lstinline{C} exits? After \lstinline{C} exits?
		How do you ensure that all resources are freed in each case?
		How about when \lstinline{P} terminates without waiting, before \lstinline{C} exits? After \lstinline{C} exits?
		Are there any special cases?
	\end{qc}
	\TODO{PCB deallocation}

\end{aspect}

\begin{aspect}{RATIONALE}
	\begin{qc}
		B9: Why did you choose to implement access to user memory from the kernel in the way that you did?
	\end{qc}
	\TODO{accessing user memory from kernel}

	\begin{qc}
		B10: What advantages or disadvantages can you see to your design for file descriptors?
	\end{qc}
	\TODO{file descriptor design}

	\begin{qc}
		B11: The default \lstinline{tid_t} to \lstinline{pid_t} mapping is the identity mapping.
		If you changed it, what advantages are there to your approach?
	\end{qc}
	\TODO{improving the thread-process mapping}
\end{aspect}
