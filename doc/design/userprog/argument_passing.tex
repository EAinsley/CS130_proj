
\section{ARGUMENT PASSING}

\begin{aspect}{DATA STRUCTURES}
	\begin{qc}
		A1: Copy here the declaration of each new or changed
		`struct' or `struct' member, global or static variable, `typedef', or enumeration.
		Identify the purpose of each in 25 words or less.
	\end{qc}
	In \lstinline{userprog/process.c}
	\begin{lstlisting}
/* Arguments for the start_process thread function.
   A `child_arg` structure is created on the stack of `process_execute`
   and the address is passed to `start_process` which will be run in a new child thread.
   Parent process only execution of `process_execute` after `sema_down(&child_arg->sema_start)` to prevent data race.
*/
struct child_arg
{
  // pointer to parent process TCB/PCB, start_process will create TCB/PCB for the child process and
  // link the TCB/PCB to the PCB of the parent process (the process that called `process_create`)
  struct thread *parent;
  // up when the child process is started or the load ELF procedure failed.
  // process_create will continue execution only when `sema_start` is up
  struct semaphore sema_start;
  // copy of the command line string: ELF file name + arguments
  char *fn_copy;               
  // whether load ELF procedure has succeeded
  bool load_failed;
};

/* helper functions: push arguments from command line onto stack (the stack frame pointer can be referenced as *esp)
   these functions are called by `start_process`.
*/
// esp: stack pointer, name: elf executable file name, args: the command line string except the first token i.e, the elf path
static void prepare_stack (void **esp, char *name, char *args);
// align the stack pointer to word address boundary by pushing zero-valued paddings.
static void esp_push_align (void **esp_);
// push a word onto the stack
static void esp_push_u32 (void **esp_, uint32_t val);
// copy a pointer onto the stack
static void esp_push_ptr (void **esp_, void *ptr);
	\end{lstlisting}
\end{aspect}

\begin{aspect}{ALGORITHMS}
	\begin{qc}
		A2: Briefly describe how you implemented argument parsing.
		How do you arrange for the elements of \lstinline{argv[]} to be in the right order?
		How do you avoid overflowing the stack page?
	\end{qc}
	We implement argument parsing with \lstinline{strtok_r} and populate the stack of the user program according to 80X86 call convention.\\

	This is done in the child thread starting function \lstinline{start_process},
	where the ELF executable file name, interrupt stack frame ESP pointer and the rest command line arguments are given.
	\begin{enumerate}
		\item Prepare a interrupt frame.
		      We will start running the user program by simulating return from interrupt.
		\item Use \lstinline{strtok_r} to find the first token of the command, that is, the ELF executable file path.
		\item Invoke \lstinline{load} on the file path obtained in last step to load the ELF executable file to memory and store the entry point address in the interrupt frame EIP register.
		      Set the \lstinline{child_arg->load_failed} flag if load failed.
		\item If the previous load step succeeded, call \lstinline{prepare_stack} to prepare the arguments.
		      \begin{enumerate}
			      \item Repeatedly call \lstinline{strtok_r} to extract arguments.
			            Each time a new argument is found, use \lstinline{esp_push_str} to copy it onto stack
			            and record its starting address.
			      \item Push the ELF executable
			      \item Call \lstinline{esp_push_align} to align the stack pointer ESP register to word boundary.
			      \item Push the pointer to each argument which are previously recorded when parsing the command line
			            onto stack with \lstinline{esp_push_ptr}.\\
			            These pointers forms the \lstinline{argv[]} array.
			      \item Push the address of \lstinline{argv[0]} (this is the current value of \lstinline{esp}) onto stack.
			            This is the pointer to the \lstinline{argv[]} array.
			      \item Finally, push the number of arguments onto stack and push a NULL pointer for the dummy return address.
		      \end{enumerate}
		\item Call semaphore up on \lstinline{child_arg->sema_start}.
	\end{enumerate}

	Stack overflow is not handled in our code.
	PintOS limits the command for \lstinline{run_task} to 128 bytes, so there is no risk of stack overflowing.
\end{aspect}

\begin{aspect}{RATIONALE}
	\begin{qc}
		A3: Why does Pintos implement \lstinline{strtok_r()} but not \lstinline{strtok()}?
	\end{qc}
	According the \lstinline{strtok(3)} man page.
	\begin{tcolorbox}
		\textbullet The \lstinline{strtok()} function uses a static buffer while parsing, so it's not thread safe.
	\end{tcolorbox}
	PintOS does not provide \lstinline{strtok()} function to eliminate the possibility of programmer making unsynchronized calls to it.

	\begin{qc}
		A4: In Pintos, the kernel separates commands into a executable name and arguments.
		In Unix-like systems, the shell does this separation.
		Identify at least two advantages of the Unix approach.
	\end{qc}
	\begin{itemize}
		\item Shell can provided more user-friendly services.
		      For example, the shell can detect and notify the user that the command can not be found or the argument sequence is incorrect.
		\item Implementing it in shell opens a door to various features: A shell can support argument string containing spaces.
		      The shell may also allow variable substituion on the command line arguments.
	\end{itemize}
\end{aspect}
