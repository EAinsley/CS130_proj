
\section{ARGUMENT PASSING}

\begin{aspect}{DATA STRUCTURES}
	\begin{qc}
		A1: Copy here the declaration of eachnew or changed
		`struct' or `struct' member, global or static variable, `typedef', or enumeration.
		Identify the purpose of each in 25 words or less.
	\end{qc}
	\TODO{paste and explain new or changes to code}
\end{aspect}

\begin{aspect}{ALGORITHMS}
	\begin{qc}
		A2: Briefly describe how you implemented argument parsing.
		How do you arrange for the elements of \lstinline{argv[]} to be in the right order?
		How do you avoid overflowing the stack page?
	\end{qc}
	\TODO{how to arrange arguments on stack according to X86 calling convention}
\end{aspect}

\begin{aspect}{RATIONALE}
	\begin{qc}
		A3: Why does Pintos implement \lstinline{strtok_r()} but not \lstinline{strtok()}?
	\end{qc}
	\TODO{\lstinline{strtok} race condition}

	\begin{qc}
		A4: In Pintos, the kernel separates commands into a executable name and arguments.
		In Unix-like systems, the shell does this separation.
		Identify at least two advantages of the Unix approach.
	\end{qc}
	\TODO{advantages of having a shell for command line arguments handling}
\end{aspect}
