\section{PRIORITY SCHEDULING}
\begin{aspect}{DATA STRUCTURES}
  \begin{qc}
    B1: Copy here the declaration of each new or changed `struct' or `struct' member, global or static variable, `typedef', or enumeration.  Identify the purpose of each in 25 words or less.
  \end{qc}
  \begin{qc}
    B2: Explain the data structure used to track priority donation.
    Use ASCII art to diagram a nested donation.  (Alternately, submit a
    .png file.)
  \end{qc}
\end{aspect}

\begin{aspect}{ALGORITHMS}
  \begin{qc}
    B3: How do you ensure that the highest priority thread waiting for a lock, semaphore, or condition variable wakes up first?
  \end{qc}

  \begin{qc}
    B4: Describe the sequence of events when a call to \lstinline{lock_acquire()} causes a priority donation.  How is nested donation handled?
  \end{qc}

  \begin{qc}
    B5: Describe the sequence of events when \lstinline{lock_release()} is called on a lock that a higher-priority thread is waiting for.
  \end{qc}

\end{aspect}

\begin{aspect}{SYNCHRONIZATION}

  \begin{qc}
    B6: Describe a potential race in \lstinline{thread_set_priority()} and explain how your implementation avoids it.  Can you use a lock to avoid this race?
  \end{qc}

\end{aspect}

\begin{aspect}{RATIONALE}
  \begin{qc}
    B7: Why did you choose this design?  In what ways is it superior to another design you considered?
  \end{qc}

\end{aspect}