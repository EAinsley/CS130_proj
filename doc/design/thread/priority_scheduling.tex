\section{PRIORITY SCHEDULING}
\begin{aspect}{DATA STRUCTURES}
	\begin{qc}
		B1: Copy here the declaration of each new or changed `struct' or `struct' member, global or static variable, `typedef', or enumeration.  Identify the purpose of each in 25 words or less.
	\end{qc}

	For \lstinline{struct thread} in \lstinline{threads/thread.h}.
	\begin{lstlisting}
    int priority, original_priority; /* Priority after/before donation */

    struct lock *wait_lock; /* the lock for which this thread is waiting for,
    NULL for running thread */
    struct list hold_locks; /* the locks hold by this thread */
        \end{lstlisting}

	For \lstinline{struct lock} in \lstinline{threads/synch.h}.
	\begin{lstlisting}
    int priority;          /* maximum waiter priority */
    struct list_elem elem; /* for `thread::hold_locks` list, see thread.c */
        \end{lstlisting}

	\begin{qc}
		B2: Explain the data structure used to track priority donation.
		Use ASCII art to diagram a nested donation.  (Alternately, submit a
		.png file.)
	\end{qc}

	The main idea of priority donation is that
	a thread get priority boost for holding a lock for which a thread of higher priority thread is waiting.
	\begin{description}
		\item[thread] maintain a list of locks hold by this thread and the lock for which this thread is waiting.
		\item[lock] maintain a list of threads that are waiting for this lock and the lock holder.
	\end{description}
	\tcbincludegraphics{pics/donation.png}

\end{aspect}

\begin{aspect}{ALGORITHMS}
	\begin{qc}
		B3: How do you ensure that the highest priority thread waiting for a lock, semaphore, or condition variable wakes up first?
	\end{qc}

	By keeping the threads waiting for a lock/semaphore/condvar in an ordered linked list sorted by their priority.
	Use insertion sort to maintain the order when a new thread enter the waiting queue or a thread's priority changes.
	Always pop and unblock the first thread in the waiting list.

	\begin{qc}
		B4: Describe the sequence of events when a call to \lstinline{lock_acquire()} causes a priority donation.
		How is nested donation handled?
	\end{qc}

	\begin{enumerate}
		\item interrupt is turned off, the current thread now has exclusive access of the single CPU core.
		\item try to acquire the lock but failed because another thread is already holding the lock.
		\item insert the current thread into the waiter list of the lock.
		\item if the current thread's priority is higher than all the other waiting threads, then priority donation is trigger.
		\item update the maximum waiter priority for that lock, if the donated priority is updated,
		      pass the priority to the thread who is holding this lock.
		\item if the priority of the lock holder changes and the lock holder is waiting for another lock,
		      then do the priority donation recursively.
		\item the recursive procedure stops when the lock waiter maximum priority can not be updated or the depth reaches some threshold,
		      the latter case indicates that a dead lock happens.
		\item interrupt level is recovered.
		\item the thread is blocked after call to \lstinline{sema_down()} and another thread gains the control of CPU.
		\item when the thread is resumed and acquired the lock,
		      add this lock to the thread's holding lock list.
		      Compute the maximum holding lock priority (a priority of a lock is the maximum waiter priority of this lock) and update the priority of this thread.
	\end{enumerate}

	\begin{qc}
		B5: Describe the sequence of events when \lstinline{lock_release()} is called on a lock that a higher-priority thread is waiting for.
	\end{qc}

	\begin{enumerate}
		\item interrupt is disable, the thread now has exclusive access of the CPU core.
		\item remove the lock from the thread's holding lock list.
		      Recompute the maximum holding lock priority and update the priority of this thread.
		      The priority donation from the threads waiting for this lock is now canceled.
		\item interrupt level is recovered.
		\item the thread calls \lstinline{sema_up()} to release the lock, some thread may occupy the CPU after that.
	\end{enumerate}

\end{aspect}

\begin{aspect}{SYNCHRONIZATION}

	\begin{qc}
		B6: Describe a potential race in \lstinline{thread_set_priority()} and explain how your implementation avoids it.
		Can you use a lock to avoid this race?
	\end{qc}

	Let's say that we have three threads t0, t1 and t2.
	\begin{itemize}
		\item t1 is to change the priority of t0 to $p$
		\item t2 is to donate priority of $q$ to t0.
		\item $p<q$
	\end{itemize}

	Consider the following execution time line.
	\begin{enumerate}
		\item t1 calls \lstinline{thread_set_priority()} and was about to set the new priority for t0.
		\item an external interrupt came and the scheduler is trigger to switch to t2.
		\item t2 donate $q$ to t0.
		\item an external interrupt came and the scheduler is trigger to switch to t1.
		\item the higher donated priority $q$ is overwritten by $p$
	\end{enumerate}
	This race condition makes the priority of t0 become lower which may cause dead lock.

	To prevent thread execution interleaving, disable the interrupt in the beginning of \lstinline{thread_set_priority()} and re-enable right before the function returns.
	Split the priority field of a thread into \lstinline{original_priority, priority_after_donation} so that donated priority won't be overwritten.\\
	For every thread, the priority donated to it is the maximum priority of the threads who are waiting for a lock hold by this thread.
	To find donated priority, use a for-loop to go through the list \lstinline{hold_locks} and compute the maximum of \lstinline{lock.waiter_max_priority}.
	\lstinline{thread_set_priority()} is also modified to changes the \lstinline{original_priority} and re-compute the \lstinline{priority_after_donation}.

\end{aspect}

\begin{aspect}{RATIONALE}
	\begin{qc}
		B7: Why did you choose this design?  In what ways is it superior to another design you considered?
	\end{qc}
	The final design that we implemented is not a straight-forward one.
	Initially,
	we came up with a brute force solution: Every time a thread acquires a lock $L$,
	we first use BFS to find the who donate priority directly or indrectly.
	Then use DFS to update the priority of the holder of $L$ denoted by $t_0$,
	the holder of the lock $t_0$ is waiting denoted by $t_1$ and
	the holder of the lock $t_1$ is waiting for.\\

	This approach is tedious, it requires us to look at the priority donors and priority receivers at the same time.
	Also, the recursive calls might be inefficient.\\

	After careful observation, we found that there is no need to track the threads donating priority to this thread.
	Priority donation and cancelation can be implemented by only updating the priority of lock holder.\\
	This approach is easy to code compared to other approaches and is has better performance since the priority donation chain is much more simpler.
\end{aspect}
