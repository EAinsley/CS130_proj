\section*{PAGING TO AND FROM DISK}

\begin{aspect}{DATA STRUCTURES}

	\begin{qc}
		B1: Copy here the declaration of each new or changed \lstinline{struct} or member,
		global or static variable, \lstinline{typedef}, or enumeration.
		Identify the purpose of each in 25 words or less.
	\end{qc}


	In \lstinline{vm/swap.h} and \lstinline{vm/swap.c}.
	\begin{lstlisting}
/* used to index the sectors of a swap-out page in SWAP block: sector [8*swap_idx,8*swap_idx +7] */
typedef uint32_t swap_idx;

static struct block *swap_block; // the block device whose ROLE is SWAP, used as swap partition
static struct bitmap *used_slots; // a bitmap used to track used/unused sectors in swap partition
static struct lock swap_lock; // prevent data race on used_slot bitmap. NOTE: block device has internal synchronization.
static const uint32_t SEC_PER_FRAME = PGSIZE / BLOCK_SECTOR_SIZE; // number of sectors used to store one page
\end{lstlisting}

In \lstinline{vm/frame.h} and \lstinline{vm/frame.c}.
	\begin{lstlisting}
struct vm_frame
{
/* A newly allocated frame is pinned so that it won't be evicted before data is loaded.
This mechanism can help us to reduce the swap ping-pong when VM is heavily loaded.
*/
bool pin;
};

static struct list frame_list;
static struct list_elem *clock_pointer;
\end{lstlisting}

In \lstinline{vm/sup_page.h} and \lstinline{vm/sup_page.c}.
	\begin{lstlisting}
struct sup_page_entry
{
/* The swap slot index in SWAP where this page is stored.
  Only applicable when status==ON_SWAP */
swap_idx swap_slot;
}
	\end{lstlisting}
\end{aspect}

\begin{aspect}{ALGORITHMS}
	\begin{qc}
		B2: When a frame is required but none is free, some frame must be
		evicted.
		Describe your code for choosing a frame to evict.
	\end{qc}
	See \lstinline{struct vm_frame *frame_get_victim (void)} function in \lstinline{vm/frame.c}.\\
	We implement \emph{clock} algorithm which choose a framem to evict in the following procedure:
	\begin{enumerate}
		\item Keep track all the frames in used with a cyclic-list structure.
		      Keep a clock pointer which can walk through the list.
		\item Proceed the clock pointer by one step.
		\item If clock pointer is pointing to a frame whose reference bit is not set in the page-directory (using \lstinline{pagedir_is_accessed})),
		      choose it as the victim.\\
		      Otherwise, clear the reference bit. We may evict this frame if no un-accessed frame is found.
	\end{enumerate}
	This algorithm is guaranteed to found a victim in $2\times f$ steps where $f$ is the number of frames in use.
	The performance is satisfiable, it can pass all the relavent test cases.

	\begin{qc}
		B3: When a process $P$ obtains a frame that was previously used by a process $Q$,
		how do you adjust the page table (and any other data structures) to reflect the frame $Q$ no longer has?
	\end{qc}
	We shall performe the following procedure atomically:
	\begin{enumerate}
		\item Find the page associated to this frame in page-directory of $Q$.
		\item Remove the mapping in page-directory.
		\item Write the frame to swap and set the \lstinline{swap_idx} in sup-table.
		\item Set the status of this page to \lstinline{IN_SWAP} in sup-table.
	\end{enumerate}
	To implement it, we disable the interrupt before performaing swap-out.

	\begin{qc}
		B4: Explain your heuristic for deciding whether a page fault for an invalid virtual address
		should cause the stack to be extended into the page that faulted.
	\end{qc}
	We use the following heuristic to infer if we need to extend stack:
	\begin{itemize}
		\item the page-fault must be a recoverable one: Not a permission violation.
		\item the fault address is above \lstinline{esp} of the user process thread.
		\item the fault address is \lstinline{esp-4} if a \lstinline{PUSH} instruction cause it.
		\item the fault address is \lstinline{esp-32} if a \lstinline{PUSHA} instruction cause it.
	\end{itemize}

\end{aspect}

\begin{aspect}{SYNCHRONIZATION}
	\begin{qc}
		B5: Explain the basics of your VM synchronization design.
		In particular, explain how it prevents deadlock.
		(Refer to the textbook for an explanation of the necessary conditions for deadlock.)
	\end{qc}
	We have mainly three modules coordinating to provide VM functionality:
	\begin{itemize}
		\item The supplemental page table in PCB of every user process.
		\item The frame resource management: Keep track of every in use frame and the page associated to it.
		\item The swap partition management: Write to or read the data from disk to frame.
	\end{itemize}
	And we have the following communications and synchronizations:
	\begin{itemize}
		\item If a process updates its own page-directory and supplemental page table, no synchronization is needed.
		\item Any call to frame resource management module will acquire the \lstinline{struct lock frame_lock} if it need to update the frames in-use list.
		\item Any call to swap partition management module will acquire the \lstinline{struct lock swap_lock} when accessing the unused-disk-sectors bitmap.
		\item When the frame management module want to evict a in-use frame:
		      Disable the interrupt and update the supplemental page table as well as the page directory.
	\end{itemize}
	We can see that no cyclic dependency exists so deadlock will never happens.

	\begin{qc}
		B6: A page fault in process $P$ can cause another process $Q$'s frame to be evicted.
		How do you ensure that $Q$ cannot access or modify the page during the eviction process?
		How do you avoid a race between $P$ evicting $Q$'s frame and $Q$ faulting the page back in?
	\end{qc}
	We disable the interrupt so $Q$ won't be able to interfere the eviction procedure (write to swap, update sup-table).\\
	The two issuses are avoided as the eviction is an atomic operation.

	\begin{qc}
		B7: Suppose a page fault in process $P$ causes a page to be read from the file system or swap.
		How do you ensure that a second process $Q$ cannot interfere
		by e.g. attempting to evict the frame while it is still being read in?
	\end{qc}
	When $P$ triggers page-fault and need to read a page back from swap,
	we allocate a frame for that page in a critical section (acquire \lstinline{struct lock frame_lock}.).
	Thus, it is impossible for two processes to get the same frame.\\

	We set the \lstinline{pin} field of the newly allocated frame and unpin it after the data is read from swap.
	Therefore, this frame will not be evicted when we are recovering it from swap. Data-integrity is ensured.

	\begin{qc}
		B8: Explain how you handle access to paged-out pages that occur during system calls.
		Do you use page faults to bring in pages (as in user programs),
		or do you have a mechanism for ``locking'' frames into physical memory,
		or do you use some other design?
		How do you gracefully handle attempted accesses to invalid virtual addresses?
	\end{qc}

	We do not change the code of syscall handlers, instead we use page-fault to bring it in.
	We save the caller's page-directory and sup-page table before calling syscall handlers so
	the page-fault in kernel space can correctly bring the page in for the user-process.
\end{aspect}


\begin{aspect}{RATIONALE}
	\begin{qc}
		B9: A single lock for the whole VM system would make synchronization easy, but limit parallelism.
		On the other hand, using many locks complicates synchronization and raises the
		possibility for deadlock but allows for high parallelism.
		Explain where your design falls along this continuum and why you chose to design it this way.
	\end{qc}
	Our design lies the the middle of the spectrum, multiple coarse-grained locks are used:
	\begin{itemize}
		\item If a process updates its own page-directory and supplemental page table, no synchronization is needed.
		\item Any call to frame resource management module will acquire the \lstinline{struct lock frame_lock} if it need to update the frames in-use list.
		\item Any call to swap partition management module will acquire the \lstinline{struct lock swap_lock} when accessing the unused-disk-sectors bitmap.
		\item When the frame management module want to evict a in-use frame:
		      Disable the interrupt and update the supplemental page table as well as the page directory.
	\end{itemize}
	The above mechanism allows more parallelism than a synchronizing all operations on VM with a global lock.
	Deadlocks are completely out of our concern as no cyclic lock dependency exists.

	Moreover this design strikes a nice balance between coding complexity and performance.
\end{aspect}
