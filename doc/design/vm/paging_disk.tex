\section*{PAGING TO AND FROM DISK}

\begin{aspect}{DATA STRUCTURES}

	\begin{qc}
		B1: Copy here the declaration of each new or changed \lstinline{struct} or member,
		global or static variable, \lstinline{typedef}, or enumeration.
		Identify the purpose of each in 25 words or less.
	\end{qc}

\end{aspect}

\begin{aspect}{ALGORITHMS}
	\begin{qc}
		B2: When a frame is required but none is free, some frame must be
		evicted.  Describe your code for choosing a frame to evict.
	\end{qc}

	\begin{qc}
		B3: When a process $P$ obtains a frame that was previously used by a process $Q$,
		how do you adjust the page table (and any other data structures) to reflect the frame $Q$ no longer has?
	\end{qc}

	\begin{qc}
		B4: Explain your heuristic for deciding whether a page fault for an invalid virtual address
		should cause the stack to be extended into the page that faulted.
	\end{qc}

\end{aspect}

\begin{aspect}{SYNCHRONIZATION}
	\begin{qc}
		B5: Explain the basics of your VM synchronization design.
		In particular, explain how it prevents deadlock.
		(Refer to the textbook for an explanation of the necessary conditions for deadlock.)
	\end{qc}

	\begin{qc}
		B6: A page fault in process $P$ can cause another process $Q$'s frame to be evicted.
		How do you ensure that $Q$ cannot access or modify the page during the eviction process?
		How do you avoid a race between $P$ evicting $Q$'s frame and $Q$ faulting the page back in?
	\end{qc}

	\begin{qc}
		B7: Suppose a page fault in process $P$ causes a page to be read from the file system or swap.
		How do you ensure that a second process $Q$ cannot interfere
		by e.g. attempting to evict the frame while it is still being read in?
	\end{qc}

	\begin{qc}
		B8: Explain how you handle access to paged-out pages that occur during system calls.
		Do you use page faults to bring in pages (as in user programs),
		or do you have a mechanism for ``locking'' frames into physical memory,
		or do you use some other design?
		How do you gracefully handle attempted accesses to invalid virtual addresses?
	\end{qc}
\end{aspect}


\begin{aspect}{RATIONALE}
	\begin{qc}
		B9: A single lock for the whole VM system would make synchronization easy, but limit parallelism.
		On the other hand, using many locks complicates synchronization and raises the
		possibility for deadlock but allows for high parallelism.
		Explain where your design falls along this continuum and why you chose to design it this way.
	\end{qc}
\end{aspect}
