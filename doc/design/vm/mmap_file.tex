\section*{MEMORY MAPPED FILES}

\begin{aspect}{DATA STRUCTURES}
	\begin{qc}
		C1: Copy here the declaration of each new or changed \lstinline{struct} or member,
		global or static variable, \lstinline{typedef}, or enumeration.
		Identify the purpose of each in 25 words or less.
	\end{qc}

	In \lstinline{threads/thread.h}
	\begin{lstlisting}
struct thread
{
  struct list mmap_list; // list used to record mmap information
};

/* one record in mmap_list: a successfully established memory-file mapping */
struct thread_mmap_node
{
  mapid_t mapid; // The allocated mapid
  void *upage_addr; // The first mapped user page address
  uint32_t pages_count; // number of pages in the mapping: ceil(file_length/page_size)
  struct list_elem list_elem; // chain the nodes in mmap_list
};
			\end{lstlisting}
	In \lstinline{vm/sup_page.c}
	\begin{lstlisting}
struct sup_page_entry
{
  /* whether this page is a memory-file mapping */
  bool mapped;
}
	\end{lstlisting}
	We also reuses the \lstinline{struct lazy_load_page} in supplemental table entry.
\end{aspect}


\begin{aspect}{ALGORITHMS}
	\begin{qc}
		C2: Describe how memory mapped files integrate into your virtual memory subsystem.
		Explain how the page fault and eviction processes differ between swap pages and other pages.
	\end{qc}
	See \lstinline{vm_sup_page_map}, \lstinline{vm_sup_page_unmap} and \lstinline{vm_sup_page_writeback}i in \lstinline{vm/sup_page.c}.\\
	Memory-mapped pages are similar to ELF lazy load pages:
	we simply add a few supplemental page table entry with \lstinline{mapped=true} for the mapped pages to record the mapped file and the offset.
	When a recoverable page-fault happens, we allocate a frame and read the contents from the mapped file.\\

	When evicting a mapped page, we write the page back to file if the dirty bit in page directory is set and clears the dirty bit.\\

	\begin{qc}
		C3: Explain how you determine whether a new file mapping overlaps any existing segment.
	\end{qc}
	When loading or extending a segment of the user process, we add a lazy load entry in the supplemental page table which will never be removed until the program exits.
	For every mapped page to be created, we check to see if a table entry already exists at that address.
\end{aspect}

\begin{aspect}{RATIONALE}
	\begin{qc}
		C4: Mappings created with \lstinline{mmap} have similar semantics to those of data demand-paged from executables,
		except that \lstinline{mmap} mappings are written back to their original files, not to swap.
		This implies that much of their implementation can be shared.
		Explain why your implementation either does or does not share much of the code for the two situations.
	\end{qc}
	Our implementation of \lstinline{mmap} reuses the code for \lstinline{load}: See \lstinline{vm/sup_page.c}.\\
	The \lstinline{vm_sup_page_map} is forwarded to \lstinline{vm_sup_page_install_files} which is used in ELF lazy load.
	We also set the table entry \lstinline{mapped} field to \lstinline{true} to indicate that this page is created by \lstinline{mmap}.\\

	We adopt the philosophy of \emph{never repeat yourself}.
	Writing similar code multiple times is a waste of time and might introduce unexpected bugs.
\end{aspect}

